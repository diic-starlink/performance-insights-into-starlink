\section{Data Collection} \label{sec:data-collection}

For analysis, we create a dataset containing measurements from various sources
about networked satellite systems. Sadly, only data from Starlink devices was
included in the thesis as other satellite network systems did not have
sufficient data openly available.

The data originates from RIPE Atlas, Cloudflare Radar, and N2YO. While
N2YO provides data about satellites, the others hold measurement results. In
the case of N2YO, the website was crawled, while the others provide API
access.

The process of data collection is illustrated in
Figure~\ref{fig:data-collection-process}.

\begin{figure}[h]
	\centering
	\includegraphics[width=0.7\textwidth]{./chapters/3-methodology/img/architecture.drawio.pdf}
	\caption{Architecture of Data Collection}
	\label{fig:data-collection-process}
\end{figure}

The data is collected from each platform and inserted into a PostgreSQL
database. The database shall allow producers to quickly insert new data (i.e.,
new rows). Transactional databases are the best choice for that, e.g.,
PostgreSQL. To quickly analyze data, an analytical database is the best choice.
For that purpose, Parquet files can be used. Therefore, the data from
PostgreSQL is dumped into the Parquet files. This creates a redundant storage,
for the sake of analysis speed.

The resulting data format is shown in Figure~\ref{fig:er-diagram}.

\begin{figure}
	\includegraphics[width=\textwidth]{./chapters/3-methodology/img/er-diagram.drawio.pdf}
	\caption{ER Diagram of Data Schema in PostgreSQL Database.}
	\label{fig:er-diagram}
\end{figure}

The database includes Ping, Traceroute, TLS, HTTP, Disconnect Event, and DNS
measurement data, as well as information about the RIPE Atlas probes and all
satellites ever launched (including rocket bodies).
The whole database comprises more than 150 GB.

Additionally, the analysis uses data from IPinfo. This data is however not
included in the dataset and has to be obtained from IPinfo itself.

\subsection*{RIPE Atlas Data}

RIPE Atlas offers various probes connected via networked satellite systems. At
the moment of writing, all of those probes are connected via Starlink.
Therefore, all the data from RIPE Atlas probes is Starlink data.

Probes are computers running the
\href{https://github.com/RIPE-NCC/ripe-atlas-software-probe}{probe software
	from RIPE Atlas}. They are centrally connected to the RIPE Atlas
servers and
can be used by any person to perform measurements against them. The possible
measurements are defined by the
\href{https://atlas.ripe.net/docs/apis/rest-api-reference/}{RIPE Atlas REST
	API}.

Overall, there are 150 probes from 26 countries. Each probe performs basic
measurements on a regular schedule (so-called built-in measurements). The
built-in measurements are the main source of data and serve as historical
record. This allows to analyze data from 2022. Even if there is Starlink data
prior to 2022, it originates from very little probes and therefore will not be
considered in this thesis to avoid unreliable data.
