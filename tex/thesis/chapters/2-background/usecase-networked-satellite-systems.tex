\section{Use Case of Networked Satellite Systems} \label{sec:usecase-networked-satellite-systems}

The internet has become a key-technology for communication in nearly every
aspect of the society. Having no access to the internet will result in
significant drawbacks. However, internet requires a complex infrastructure with
high cost. Especially in distant location with little population, building such
an infrastructure will not be affordable.

Therefore, people came up with the idea of using satellites to communicate with
distant locations. In theory, only three satellites are required to communicate
with any point on earth (except for polar regions)
\cite{DBLP:conf/5gwf/HofmannK19}. The cost of providing this number of
satellites is much less compared to the costs of providing a terrestrial
network infrastructure with similar accessibility.

There is a group of people that will likely never receive terrestrial network
infrastructure: people on boats and planes. While there is cellular internet
access, it is not offered on the sea and in the high sky. Therefore, the only
option of internet access is satellite internet, which is served all over the
world.

Additionally, networked satellite systems are much more resilient to physical
influences like earthquakes, terrorist attacks, or storms
\cite{DBLP:conf/pam/StevensIBD24}. Terrestrial infrastructure can be destroyed
easily and therefore especially governments contracted networked satellite
system ISPs. A prominent example is the use of Starlink in conflict zones like
Ukraine.
