\section{Usecase of Networked Satellite Systems} \label{sec:usecase-networked-satellite-systems}

Networked satellite systems are around for quite a while since humanity posses
both satellites and the internet. The internet has become a key-technology for
communication in nearly every aspect of the society. Having no access to the
internet will result in significant drawbacks. However, internet requires a
complex infrastructure with high cost. Especially in distant location with
little population, building such an infrastructure will not be affordable.

% TODO: Add citation for the statement "only three satellites are required"
Therefore, people came up with the idea of using satellites to communicate with
distant locations. In theory, only three satellites are required to communicate
with any point on earth. The cost of providing this number of satellites is
much less compared to the costs of providing a terrestrial network
infrastructure with similar accessibility.

There is one group of people that will not receive terrestrial network
infrastructure: people on boats and planes. While there is cellular internet
access, it is not offered on the sea and in the high sky. Therefore, the only
option of internet access is satellite internet, which is served all over the
world.

Additionally, networked satellite systems are much more resilient to physical
influences like earthquakes, terrorist attacks, or storms
\cite{DBLP:conf/pam/StevensIBD24}. Terrestrial infrastructure can be destroyed
easily and therefore especially governments contracted networked satellite
system ISPs.


