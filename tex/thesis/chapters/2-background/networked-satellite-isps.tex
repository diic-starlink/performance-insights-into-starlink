\section{Networked Satellite ISPs} \label{sec>isps}

There are a couple of ISPs out there providing internet access via satellites.

\begin{table}[]
	\caption{Different networked satellite ISPs}
	\label{fig:satellite-isp}
	\begin{tabular}{lrr}
		\toprule
		ISP       & Category & Customer Group      \\
		\midrule
		Starlink  & LEO      & Private \& Business \\
		OneWeb    & LEO      & Business            \\
		HughesNet & GEO      & Private \& Business \\
		Intelsat  & GEO      & Business            \\
		Viasat    & GEO      & Business            \\
		Orbcomm   & GEO      & Business            \\
		Iridium   & GEO      & Private \& Business \\
		\bottomrule
	\end{tabular}
\end{table}

Table~\ref{fig:satellite-isp} shows different ISPs for networked satellite systems.
The differentiate mostly by providing a \ac{LEO} or \ac{GEO} service. The only \ac{LEO}
ISPs are Starlink and OneWeb. Starlink has an economical advantage over most of its competitors
as they are attached to SpaceX. SpaceX is a company providing spacecraft manufacturing and launch services.
This makes the launch of satellites less tedious compared to Starlink's competitors.

Kohnmann \cite{Kohnmann24} reported about intentions of Amazon launching a networked satellite provider called Kuiper. It is planned to be launched by the end of 2026.
