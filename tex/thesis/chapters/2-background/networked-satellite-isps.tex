\section{Satellite Network Operators} \label{sec>isps}

There are a couple of \ac{SNO}s out there providing internet access via
satellite network. Table~\ref{fig:satellite-isp} shows different \ac{SNO}s for
networked satellite systems.

\begin{table}[]
	\caption{Different networked satellite ISPs}
	\label{fig:satellite-isp}
	\begin{tabular}{lrr}
		\toprule
		ISP       & Category & Customer Group      \\
		\midrule
		Starlink  & LEO      & Private \& Business \\
		OneWeb    & LEO      & Business            \\
		HughesNet & GEO      & Private \& Business \\
		Intelsat  & GEO      & Business            \\
		Viasat    & GEO      & Business            \\
		Orbcomm   & GEO      & Business            \\
		Iridium   & GEO      & Private \& Business \\
		\bottomrule
	\end{tabular}
\end{table}

They differentiate mostly by providing either a \ac{LEO} or \ac{GEO} service.
The only \ac{LEO} \ac{SNO}s are Starlink and OneWeb. Starlink has an economical
advantage over most of its competitors as they are a sub-company of SpaceX.
SpaceX is a company providing spacecraft manufacturing and launch services.
This makes the launch of satellites less tedious compared to Starlink's
competitors.

Kohnmann \cite{Kohnmann24} reported about Amazon's intentions of launching a
\ac{SNO} called Kuiper. It is planned to be launched by the end of 2026, after
two test satellites were already launched
(\href{https://www.n2yo.com/satellite/?s=58014}{KUIPER-P2}, ID58013 and
\href{https://www.n2yo.com/satellite/?s=58013}{KUIPER-P1}, ID58014) on
October~6,~2023. However, there is no data yet available.
