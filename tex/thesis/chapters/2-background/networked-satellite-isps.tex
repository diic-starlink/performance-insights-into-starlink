\section{Satellite Network Operators} \label{sec>isps}

There are a couple of \ac{SNO} out there providing internet access via
satellite network.

\begin{table}[]
	\caption{Different networked satellite ISPs}
	\label{fig:satellite-isp}
	\begin{tabular}{lrr}
		\toprule
		ISP       & Category & Customer Group      \\
		\midrule
		Starlink  & LEO      & Private \& Business \\
		OneWeb    & LEO      & Business            \\
		HughesNet & GEO      & Private \& Business \\
		Intelsat  & GEO      & Business            \\
		Viasat    & GEO      & Business            \\
		Orbcomm   & GEO      & Business            \\
		Iridium   & GEO      & Private \& Business \\
		\bottomrule
	\end{tabular}
\end{table}

Table~\ref{fig:satellite-isp} shows different \ac{SNO} for networked satellite
systems. They differentiate mostly by providing either a \ac{LEO} or \ac{GEO}
service. The only \ac{LEO} \ac{SNO}s are Starlink and OneWeb. Starlink has an
economical advantage over most of its competitors as they are a subcompany of
SpaceX. SpaceX is a company providing spacecraft manufacturing and launch
services. This makes the launch of satellites less tedious compared to
Starlink's competitors.

Kohnmann \cite{Kohnmann24} reported about intentions of Amazon launching a
networked satellite provider called Kuiper. It is planned to be launched by the
end of 2026, after two test satellites were already launched. However, there is
no data yet availalbe.
