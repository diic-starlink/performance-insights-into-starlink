\section{Satellite Network Simulators} \label{sec:satellite_network_simulators}

Amongst concrete measurements, one can also simulate networked satellite
systems. This became increasingly interesting when the constellations were
composed of many more satellites compared to traditional \ac{GEO} satellite
constellations. For example, \ac{LEO} constellations comprise hundreds to
thousands of satellites, which implies a highly complex system.

Sadly, measurements are often highly difficult as they either require acquiring
satellite hardware or recruiting users that already posses the required
hardware. Simulation would tackle both problems, while maintaining low cost. To
the best of our knowledge, we found two networked satellite simulators for
\ac{LEO} constellations.

\subsection{StarPerf} \label{sec:starperf}

\textit{StarPerf}\footnote{\href{https://github.com/SpaceNetLab/StarPerf\_Simulator}{SpaceNetLab/StarPerf\_Simulator}}
\cite{DBLP:conf/icnp/LaiLL20} is a mega-constellation performance simulation
platform. It specifically aims at measuring the impact of the movements of
satellites. Also, it measures performance in different areas. However, setting
it up required, amongst others, Matlab and STK. This made the project difficult
and expensive to test. Therefore, we did not advance in trying out
\textit{StarPerf}.

\subsection{Hypatia} \label{sec:hypatia}

\textit{Hypatia}\footnote{\href{https://github.com/snkas/hypatia}{snkas/hypatia}}
\cite{DBLP:conf/imc/KassingBASS20} is another LEO network simulation framework,
released in 2020 just like \textit{StarPerf}. It aims at a low-level simulation
on packet-level and visualizes the data. Unlike \textit{StarPerf}, it only
requires a Python3 installation. Sadly, running simulations with Hypatia is
highly complex as it requires the user to define, amongst others, the
satellites, ground stations, and points of presence. This information is hardly
available, which renders the simulations barely usable. Also Hypatia is not
maintained anymore.

\subsection{Problems of Network Simulators}

To the best of our knowledge, research has stopped relying on simulators since
2020. There is proper hardware available that allows testing in the real world,
even if it is highly expensive depending on the operator. Testing in the real
world has the advantage as it takes more variables into account. Crucial
factors for the performance of a networked satellite system are the weather,
congestion, solar magnetic storms, material failure, and many more. Those
cannot be ideally tested with simulations and will eventually produce wrong
results. Therefore, for further research of this thesis, simulations will not
be used.
