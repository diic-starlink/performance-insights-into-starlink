\section{Thesis Outline}

This thesis is structured as follows.

Chapter \ref{sec:introduction} introduces the topic and outlines the thesis. We
define the research questions that we want to solve with the thesis. Also, we
present related work that has already researched properties of \ac{SNOs}.

Chapter \ref{sec:background} explains the basic concepts of satellite
communication. It explains the schematics of networking in a satellite
constellation, but does not explain the physical layer of the technology.
Additionally, it outlines the use-cases of satellite communication together
with the companies that took on the challenge.

Advancing to the results part of the thesis, we will first introduce our
methodology. Chapter~\ref{sec:methodology} explains how we obtained our
measurement results, by accessing them on RIPE~Atlas, Cloudflare~Radar, and
N2YO. Further platforms offering more data are OONI and M-Lab. However, due to
time constraints they were not explored in this thesis.

Chapter~\ref{sec:results} analyses the obtained data. It looks at various
metrics like packet loss, latency, disconnect events, and more. We also display
historical records, since the beginning of 2022, showing a clear development.
Additionally, we differentiate in countries for the most part, where we see
significant geographical differences.

In the end, we conclude in Chapter~\ref{sec:conclusion}. We also outline where
additional work is required due to constraints, like time and money.

