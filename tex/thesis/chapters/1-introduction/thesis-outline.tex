\section{Thesis Outline}

This thesis is structured as follows.

Chapter \ref{sec:introduction} introduces the topic and outlines the thesis. We
define the research questions that we want to solve with the thesis. Also, we
present related work that has already researched properties of \ac{SNO}s.

Chapter \ref{sec:background} talks about the basic concepts of satellite
communication. It explains the schematics of networking in a satellite
constellation, but does not explain the physical layer of the technology.
Additionally, it outlines the use-cases of satellite communication and lists
companies being active in the field.

Advancing to the results part of the thesis, we will first introduce our
methodology. Chapter~\ref{sec:methodology} explains how we obtained our
measurement results, by accessing them on RIPE~Atlas, Cloudflare~Radar, and
N2YO. Further platforms offering more data are OONI and M-Lab. However, due to
time constraints they were not explored in this thesis.

Chapter~\ref{sec:results} analyses the measurements results. It looks at
various metrics like packet loss, latency, disconnect events, and more. We also
display historical records, since the beginning of 2022 to July 2024, showing a
clear development. Additionally, we differentiate in countries for the most
part, where we see significant geographical differences.

In the end, we conclude in Chapter~\ref{sec:conclusion}. We answer the research
questions to the most part with the results from the previous chapter and
outline where additional work is required due to constraints, like time and
money.
