\section{Research Questions} \label{sec:research-questions}

In this thesis, we scope for the resilience and performance of networked
satellite systems. Specially, we focus on the Starlink constellation as it is
the only system that has public data available on different measurement
platforms. Other constellations like Orbcomm or OneWeb have no or insufficient
data available and can therefore not be explored. We tried to get our
hands on a satellite system to deploy at our chair, but were not able to get a
proper response by the operators. Therefore, we solve the following research
question by the data that is openly available on websites like RIPE~Atlas or
Cloudflare~Radar.

\begin{rqbox}{1: How do networked satellite systems perform in terms of latency
		and packet loss?}
	Packet loss and latency are key factors for the performance of a
	network. Recent research stated high packet loss values, and mostly
	idealized latency values. This thesis investigates the two
	performance characteristics. We focus on the Starlink constellation as
	it is the only system that has public data available on different
	measurement platforms

	To solve this research question, we will use RIPE~Atlas and Cloudflare
	Radar.
\end{rqbox}

\begin{rqbox}{2: Do latency and packet loss correlate?}
	Rising packet losses are expected to correlate with a rising latency as
	they need to be retransmitted. It is in question whether we can observe
	such a correlation for satellite networks.

	If we cannot observe a correlation, what factors might influence the
	correlation? Amongst others, we look at factors like the intensity of
	solar magnetic storms.

	To answer this research question, we use the data obtained while
	investigating RQ~1 (i.e., data from RIPE Atlas and Cloudflare Radar).
\end{rqbox}

\begin{rqbox}{3: How do networked satellite systems route?}
	Networked satellite systems face a severe challenge when routing
	packets as they need to route through a highly complex satellite mesh.
	It includes linking satellites to one another and predict their
	performance. This thesis takes a look at the routing performance.

	To answer this research question, we use traceroute data from
	RIPE~Atlas.
\end{rqbox}
