\section{Research Questions} \label{sec:research-questions}

In this thesis, we scope for the Resilience and Performance of Networked
Satellite Systems. Specially, we focus on the Starlink constellation as it is
the only system that has public data available on different measurement
platforms. Other constellations like Orbcomm or OneWeb had no or insufficient
data available and could therefore also not be explored. We even tried to get
our hands on a satellite system to deploy at our chair, but were not able to
get a proper response by the operators. Therefore, we will solve the following
research question by the data that is openly available on websites like
RIPE~Atlas or Cloudflare~Radar.

\begin{rqbox}{1: How do networked satellite systems perform in terms of latency
		and packet loss?}
	Packet loss and latency are the main factors for the performance of a
	network. Recent research stated high packet loss values, but mostly
	idealized latency values. This thesis shall take a look at those two
	performance characteristics.

	To solve this research question, we will use RIPE~Atlas and Cloudflare
	Radar. However, RIPE~Atlas will provide the majority of the data as the
	data is more fine-grained\footnote{Cloudflare~Radar also provides
		fine-grained data, but our data collection did not collect that
		data}.
\end{rqbox}

\begin{rqbox}{2: Do latency and packet loss correlate?}
	It is expected that a rising latency correlates with a rising packet
	loss and vice versa due to the need of retransmits.
	It is in question whether this is the case for networked satellite
	systems.

	If it appears that there is no clear correlation observable, what
	factors might influence the correlation? Amongst other, we look at
	factors like the Kp index of solar magnetic storms.

	To solve this research question, we will use the data obtained from
	RQ~1 (i.e., data from RIPE Atlas and Cloudflare Radar).
\end{rqbox}

\begin{rqbox}{3: How do networked satellite systems route?}
	Networked satellite systems face a severe challenge when routing a
	packet as they need to route through a highly complex satellite system.
	It includes linking satellites to one another and predict their
	performance. This thesis takes a look at the routing performance.

	To solve this research question, we will use traceroute data from
	RIPE~Atlas.
\end{rqbox}
