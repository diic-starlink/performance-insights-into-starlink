\section{Motivation} \label{sec:motivation}

The internet plays a major role in our everyday's life. There are many
technologies available that allow to connect to the web addressing different
use cases (e.g., Wi-Fi for local area networks \cite{Henry2002} or \ac{FTTH}
for high-speed wired connections to an end user's home \cite{Aleksic2010}).

However, there are several challenges with such a setup. First, it requires
wired infrastructure, which is quite expensive for networks in remote areas.
For single end users, it is not affordable. Second, should the wired infrastructure be in place, it is vulnerable to attacks or natural catastrophes that render the service unusable.

Therefore, people came up with the idea of using satellites for communication.
The idea is not new as satellites are used for communication for quite a while
(e.g., radio signals in the 1970s \cite{Davies1980}). Satellites are resilient
to human attacks or natural catastrophes as satellites are largely inaccessible
in space. Therefore, governments installed dedicated satellite constellations
to maintain communication in any given scenario. Prominent examples of
governmental satellite constellations are \textit{Beidou} and \textit{Galileo}.

Aside from crisis intervention, also the private sector discovered the
opportunities of satellite communication. Providing communication over
satellite allows users mobile access to information that usually requires
complex infrastructure. Users can access services like geographical data, radio
frequencies, and even the internet. Especially the demand for web access is
growing, while \ac{SNO}s failed at providing acceptable latencies in the past.

However, also the demand for low latency access grew. Practice showed that
\ac{GEO} satellites (35'786 km altitude) only provided the most basic service
necessary to operator communication. However, \ac{LEO} satellites (200 --
1200~km altitude) were able to provide competitive latencies comparing with
current terrestrial access. On the other hand side, building a \ac{LEO}
satellite constellation is highly expensive, as it requires hundreds to
thousands of satellites around the global. At the moment of writing, there are
only two companies taking on this challenge: OneWeb and Starlink. Amazon also
plans to construct their own satellite constellation Kuiper, but it is not
clear whether this will become reality in the near future.

Aside from many earlier business failures \cite{Chan2002, Barboza2000}, it is
still in question how well networked satellite systems actually work and
whether they offer practical advantages. Previous work reported competitive
latencies for \ac{LEO} systems, accompanied by increased packet loss
\cite{DBLP:conf/imc/MichelTGB22}. Therefore, we will explore the performance
and resilience of \ac{SNO}s. Also, we will try to determine the correlation
between individual properties, e.g., a correlation between latency and packet
loss.
