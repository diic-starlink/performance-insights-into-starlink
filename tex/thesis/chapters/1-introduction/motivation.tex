\section{Motivation} \label{sec:motivation}

Accessing the internet works by a variety of technologies (e.g., Wi-Fi for
local area networks \cite{Henry2002} or \ac{FTTH} for high-speed wired
connections to an end user's home \cite{Aleksic2010}) addressing different use
cases.

However, there are several challenges with such a setup. First, it requires
wired infrastructure, which is quite expensive for networks in remote areas as
small number of users is available. For single end users, it is not affordable.
Second, should the wired infrastructure be in place, it is vulnerable to
attacks (e.g., cyber attacks, war, or vandalism) or natural catastrophes that
render the service unusable.

Therefore, people came up with the idea of using satellites for internet
access. The idea is not new as satellites are used for communication for quite
a while (e.g., radio signals in the 1970s \cite{Davies1980}). Satellites are
resilient to external influences as satellites are largely inaccessible in
space. Therefore, governments installed dedicated satellite constellations to
maintain communication in any given scenario. Prominent examples of
governmental satellite constellations are \textit{Beidou}, \textit{Skylo}, and
\textit{Galileo}. Aside from crisis intervention, also the private sector
discovered the opportunities of satellite communication. Providing
communication over satellite allows users mobile information access from
anywhere in the world. Users can access services like geographical data, radio
frequencies, and the internet. Especially the demand for web access is growing,
while \ac{SNO}s failed at providing a stable and low-cost
service\cite{Barboza2000, Chan2002}.

In the last years, the demand for low latency access grew. Practice showed that
\ac{GEO} satellites (35'786 km altitude) only provided the most basic service
necessary to operate communication\cite{DBLP:journals/pacmnet/RamanVCSZ23}. At the same time, \ac{LEO} satellites (200
-- 1200~km altitude) were able to provide competitive latencies comparing to
current terrestrial communication\cite{DBLP:conf/infocom/MaCZCML23}. On the other hand side, building a \ac{LEO}
satellite constellation is highly expensive, as it requires hundreds to
thousands of satellites around the globe. Currently, there are two companies
taking on this challenge: \textit{OneWeb} and
\textit{Starlink}\footnote{Starlink is a sub-company of SpaceX}.
Kuiper\footnote{Kuiper is a sub-company of Amazon.} also plans to launch a
satellite constellation, but it is not clear whether this will become reality
in the near future\cite{Kohnmann24}.

Aside from many earlier business failures \cite{Chan2002, Barboza2000}, it is
still in question how well networked satellite systems actually work and
whether they offer practical advantages. Previous work reported competitive
latencies for \ac{LEO} systems, accompanied by increased packet loss
\cite{DBLP:conf/imc/MichelTGB22}. Also, satellite constellations are expected
to be predictable. It is known what hardware is used and which course each
satellite takes. However, their performance still varies a lot (i.e., they seem
not be resilient). Therefore, we will explore the performance and resilience of
\ac{SNO}s by looking at latency, packet loss, routing data, and more.
