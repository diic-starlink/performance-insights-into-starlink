\subsection{Solarmagnetic Storms} \label{sec:solarmagnetic-storms}

Ma et al. \cite{DBLP:conf/infocom/MaCZCML23} claimed solarmagnetic storms have
a significant impact on the performance of Starlink. To check on that, we
conducted a study to see an actual correlation. We used the TLS latency data we
aquired from RIPE Atlas to correlate it with the intensity of solar magnetic
storms. Solar magnetic storms are usually identified by the Kp index
\cite{Bartels1957}. It is a number between zero and nine, where nine indicates
the strongest kind of storm. We found data historic Kp indexes on the website
of the Helmholtz-Zentrum Potsdam Deutsches GeoForschungsZentrum, G. F. Z.
\cite{GFZ2023}.

To get an understanding of the influences of solarmagnetic storms on Starlink
performance, we will look at the intensity of the solarmagnetic storm in
comparison to the TLS latency. We expect that with increasing intensity of
solarmagnetic storms (i.e., higher Kp index), the latency will also increase.
Therefore, we used the average Kp index from GFZ database over a single day and
correlated it with the mean TLS latency over each single day observed in our
dataset. The following correlation values resulted:

\begin{itemize}
	\item Pearson Correlation: $0.03$
	\item Kendall Correlation: $0.01$
	\item Spearman Correlation: $0.01$
\end{itemize}

One can see that the values are nearly zero. That implies that the dimensions
are nearly orthogonal. Therefore, we were not able to observe a correlation
between latency and Kp index. In its extreme, the data does not correlate at
all and observations form Ma~et~al.~\cite{DBLP:conf/infocom/MaCZCML23} are most
likely incorrect.

\begin{takeaway}{Correlation of Starlink Latency and Solar Magnetic Storms}
	Starlink latency and solar magnetic storms do not correlate.
\end{takeaway}

