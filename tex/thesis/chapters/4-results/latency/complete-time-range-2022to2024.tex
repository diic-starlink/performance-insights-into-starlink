\subsection{Latency over whole Data Range} \label{sec:latency-wholerange}

The performance of Starlink latency has changed over time. We looked at RIPE
Atlas TLS data that has been collected by built-in measurements (i.e.,
measurements that are continuously running in each individual probe in a fixed
time interval).

\begin{figure}
	\centering
	\begin{subfigure}[b]{0.47\linewidth}
		\includegraphics[width=\linewidth]{chapters/4-results/latency/img/latency_2022_to_2024_Germany.pdf}
		\caption{Germany}
	\end{subfigure}
	\begin{subfigure}[b]{0.47\linewidth}
		\includegraphics[width=\linewidth]{chapters/4-results/latency/img/latency_2022_to_2024_United States.pdf}
		\caption{USA}
	\end{subfigure}
	\begin{subfigure}[b]{0.47\linewidth}
		\includegraphics[width=\linewidth]{chapters/4-results/latency/img/latency_2022_to_2024_Poland.pdf}
		\caption{Poland}
	\end{subfigure}
	\begin{subfigure}[b]{0.47\linewidth}
		\includegraphics[width=\linewidth]{chapters/4-results/latency/img/latency_2022_to_2024_Austria.pdf}
		\caption{Austria}
	\end{subfigure}
	\caption{Latency History for Germany, the USA, Poland, and Austria in the period 01/2022 to 06/2024}
	\label{fig:latency_wholerange}
\end{figure}

Figure~\ref{fig:latency_wholerange} shows the history of median latencies from
January~2022 until June~2024 for Germany, the United States, Poland, and
Austria\footnote{These countries were chosen due to the completeness of data}.

One can see that the median latency is usually at around 100 to 150~ms for most
countries. However, in 2022 the latency was lower compared to 2023. Most of the
countries observed show an upwards trend in the late months of 2022 (most of
the time in December). In the late months of 2023, the latency starts to
decline again. In the last couple of months, we observe an increase in latency
once again. We assume the reason for a rise of latency in the congestion of the
Starlink network. In the recent time Starlink extended their availability
across more countries allowing more users, especially those in countries with
less networking infrastructure, to access the network. On the contrary,
Starlink also launched more satellites (2022: 3481, 2024: 6000+) and added more
\ac{PoP} which likely reduces the congestion.

\begin{figure}
	\centering
	\begin{subfigure}[t]{0.47\linewidth}
		\includegraphics[width=\linewidth]{chapters/4-results/latency/img/heatmap-median-latencies-2024.pdf}
		\caption{Median}
	\end{subfigure}
	\begin{subfigure}[t]{0.47\linewidth}
		\includegraphics[width=\linewidth]{chapters/4-results/latency/img/heatmap-average-latencies-2024.pdf}
		\caption{Average}
	\end{subfigure}
	\caption{Heatmap of Median and Average Latencies in 2024 in Europe.}
	\label{fig:heatmap-latencies-europe}
\end{figure}

Figure~\ref{fig:heatmap-latencies-europe} shows the median and average
latencies in European countries. It shows that the north west of Europe
experiences the best latencies, likely due to the presence of various \ac{GS}s.
The southern and eastern european countries experience worse latencies.
Especially Greece has a high median latency. A cause could be the absence of
\ac{GS}s in the east-european region. Italy on the other hand side experiences
a high average latency, even with \ac{GS}s being present in the country.

Looking at the CDF of Canada results in a similar observation.
Figure~\ref{fig:latency-cdfs-canada} illustrates the CDF plots for 2022 -- 2024
for Canada. We observe similar results for other countries, but only chose
Canada as it holds sufficient data for a conclusion.a

\begin{figure}
	\centering
	\begin{subfigure}[b]{0.3\linewidth}
		\includegraphics[width=\linewidth]{chapters/4-results/latency/img/cdf_latencies_in_2022_of_starlink_probes_in_canada.pdf}
		\caption{2022}
	\end{subfigure}
	\begin{subfigure}[b]{0.3\linewidth}
		\includegraphics[width=\linewidth]{chapters/4-results/latency/img/cdf_latencies_in_2023_of_starlink_probes_in_canada.pdf}
		\caption{2023}
	\end{subfigure}
	\begin{subfigure}[b]{0.3\linewidth}
		\includegraphics[width=\linewidth]{chapters/4-results/latency/img/cdf_latencies_in_2024_of_starlink_probes_in_canada.pdf}
		\caption{2024}
	\end{subfigure}
	\caption{CDF of Starlink Latencies each year in Canada.}
	\label{fig:latency-cdfs-canada}
\end{figure}

We observe and similar performance of 2022 and 2024, but a drop in performance
in 2023, similar to the conclusion we drew before.

Additionally, we conclude that approximately half of the measurement results
are below 100 ms, while the other half moves above 100 ms. This opposes
research suggesting Starlink performance is mostly in the sub-100 ms area
\cite{DBLP:conf/www/MohanFCBRMO24, DBLP:conf/icnp/LaiLL20,
	DBLP:journals/pacmnet/RamanVCSZ23, DBLP:conf/imc/MichelTGB22}.

The CDF has is continuous, up to a specific point, where it flattens, followed
by a stronger increase once again. This is similarly observed for curves of
other countries (e.g., for the USA in Figure~\ref{fig:latency-cdfs-usa}). This
observation suggests that there is a region of latencies Starlink does not
serve equally to other latencies. The specific location of the flattening
behavior varies between countries, but is usually located between 150 and 250
ms.

\begin{figure}
	\centering
	\begin{subfigure}[b]{0.3\linewidth}
		\includegraphics[width=\linewidth]{chapters/4-results/latency/img/cdf_latencies_in_2022_of_starlink_probes_in_united_states.pdf}
		\caption{2022}
	\end{subfigure}
	\begin{subfigure}[b]{0.3\linewidth}
		\includegraphics[width=\linewidth]{chapters/4-results/latency/img/cdf_latencies_in_2023_of_starlink_probes_in_united_states.pdf}
		\caption{2023}
	\end{subfigure}
	\begin{subfigure}[b]{0.3\linewidth}
		\includegraphics[width=\linewidth]{chapters/4-results/latency/img/cdf_latencies_in_2024_of_starlink_probes_in_united_states.pdf}
		\caption{2024}
	\end{subfigure}
	\caption{CDF of Starlink Latencies each year in the USA.}
	\label{fig:latency-cdfs-usa}
\end{figure}

\begin{figure}
	\centering
	\begin{subfigure}[b]{0.3\linewidth}
		\includegraphics[width=\linewidth]{chapters/4-results/latency/img/histogram_of_latencies_in_2022_of_starlink_probes_in_united_states.pdf}
		\caption{2022}
	\end{subfigure}
	\begin{subfigure}[b]{0.3\linewidth}
		\includegraphics[width=\linewidth]{chapters/4-results/latency/img/histogram_of_latencies_in_2023_of_starlink_probes_in_united_states.pdf}
		\caption{2023}
	\end{subfigure}
	\begin{subfigure}[b]{0.3\linewidth}
		\includegraphics[width=\linewidth]{chapters/4-results/latency/img/histogram_of_latencies_in_2024_of_starlink_probes_in_united_states.pdf}
		\caption{2024}
	\end{subfigure}
	\caption{EquiWidth Histogram of Starlink Latencies 2022, 2023, and 2024.}
	\label{fig:latency-histogram-usa}
\end{figure}

Analyzing the flattening behavior in more detail, we used an EquiWidth
histogram\footnote{A histogram that puts all values in a pre-defined number of
	bins, where all bins cover an equally wide data-range.} to plot the
most frequent latencies. Figure~\ref{fig:latency-histogram-usa} shows the
histogram for the USA. It becomes apparent that there is actually a major gap
between the latencies. This gap appears for every country, for every year
analyzed. It does not matter how large the country.

As this behavior appears to be independent of time and location, it must
originate from Starlink itself. As we lack the possibility to measure other
networked satellite ISPs, we do not know whether this is a general property of
networked satellite internet.

