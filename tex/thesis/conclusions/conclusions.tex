\chapter{Conclusions \& Outlook} \label{sec:conclusion}

We have looked at the performance and resilience of the Starlink satellite
system. The main research questions are listed in
Chapter~\ref{sec:research-questions}.

To answer research question 1, we looked at the TLS handshake latency in each
years 2022~--~2024 and found that Starlink latency is at approximately 80~ms
median in 2024. However, it has to be noted that Starlink latency improved
since 2022 in average, median, and minimal latency (see
Table~\ref{fig:weekday-statistics}).

We observed a strange behavior in the latency that we call the
"two-bell-pattern." It is visible in Figure~\ref{fig:latency-histogram-1} and
\ref{fig:latency-histogram-2}. It describes the behavior of serving two ranges
of latencies especially well. First lower latencies ($\approx$~80~--100~ms) and
second higher latencies ($\approx$~150~--250~ms) with a major gap in between.
The cause of this pattern is unclear.

Additionally, we wanted to find out about diurnal variation. Looking at data
from April~2024, we found that there is no difference between the individual
week days (see Figure~\ref{fig:latencies-per-weekday}). However, there is a
strong variation between the hours of the day. For example, one can expect
lower latencies at night compared to during lunchtime.

We also looked at packet loss and noted a high variation by country. Countries
such as the Philippines have packet loss ratios of up to 18\%, while Czechia
and Chile achieve less than 0.25\%. On the other hand side, highly modernized
countries like Germany and the Netherlands experience high packet loss ratios
(see Table~\ref{fig:packetloss-total}). Overall, most countries have a packet
loss of 1 to 4\%.

We expected a correlation in packet loss and latency and therefore correlated
them. We split the values by year and by country.
Table~\ref{fig:packetloss-latency-correlation} shows the resulting correlation
values, where 0 indicates orthogonal random variables and 1 or -1 indicates a
strong correlation. Some countries hold values that allow to draw a conclusion,
but others do not. We assumed that the number of probes played a role, but were
able to eliminate it as a determinant of behavior (see
Table~\ref{fig:packetloss-latency-number-probes-correlation}). The correlation
might be related to currently unknown contributing factors that need to be
taken into consideration before drawing a conclusion. Therefore, we cannot make
a statement about the correlation of latency and packet loss. Research question
2 can consequently not be answered by our research.

To answer research question 3, we further looked at the routing of the Starlink
satellite system. First, we wanted to find out the number of hops a route
usually takes. We found that most routes take 10 to 14 hops (see
Figure~\ref{fig:hops-per-measument}). The histogram for the hops also shows the
mentioned "two-bell-pattern." Both observations might be related, which was not
further researched in this thesis.

We also observed the change in latency per hop (i.e., what the traceroute
measurement latency reports in each hop). We found that usually there is a
spike in latency in the second hop. Likely, the spike originates from routing
through the Starlink satellite constellation, which is not reflected in the
traceroute results (see Figures~\ref{fig:latency-change-per-hop-1} and
\ref{fig:latency-change-per-hop-2}). We also found consequent spikes in
latency. However, in mapping the hops to their predominant ASN, we found that
those spikes lay outside the Starlink domain. Starlink has an opportunity of
improving its performance at this point, but likely other entities are actually
responsible for the spike.

Finally, we looked at the influence of solar magnetic storms on the Starlink
latency and found that there is no correlation (i.e., both random
variables are orthogonal).

\section{Future Work}

During our work, we faced several limitations that required more effort. The
first limitation were the probes that we did not hold ourselves, but which
originate from RIPE~Atlas. This means unknown people in the world set them up
as they wished. This might have corrupted the data. Ideally, the probes are
operated by the researchers themselves, but doing this a costly and tedious
undertaking.

Further, some directions were left unexplored due to time constraints. The
two-bell-pattern is an interesting observation that needs confirmation and
explanation as it was not clear why it occurred. It should also be correlated
with the similar observation in number of hops.

A detailed look at how Starlink routes packets on its satellite constellation
and later in terrestrial internet is left out. It is clear that Starlink cannot
give away complete responsibility after a packet is handed to a different AS.
At the same time, a proper model for such a routing behavior is missing.
Certain parts, like the placing of ground stations, are known, but a complete
model was not yet found.

Mentioned in Chapter~\ref{sec:bent-pipes}, \ac{SNO}s largely rely on \ac{ISL}s.
Looking at the traceroute data, it is obvious that \ac{ISL}s are used for
supporting remote regions. In parallel, research discusses the impact of
\ac{ISL}s on the performance. Their impact and ideal usage are questions that
need to be addressed and are a matter of future research.
