\chapter{Conclusions \& Outlook} \label{sec:conclusion}

We have looked at the performance and resilience of the Starlink satellite
system. Specifically, we focused on analyzing historic data in the timeframe
January~2022 to June~2024. RIPE~Atlas and Cloudflare~Radar provided TLS
handshake latency, ping, traceroute, and other measurement results.

\subsubsection*{RQ 1: How do networked satellite systems perform in terms of latency
	and packet loss?}

To answer the first research question, we looked at the TLS handshake latency and found
that Starlink latency is at approximately 80~ms median in 2024. However, it has
to be noted that Starlink latency improved since 2022 in average, median, and
minimal latency. In 2023, Starlink improved on achieve a best minimal latency
($\approx$~20ms), while being worse in media latency ($\approx$~10ms worse). In
2024, Starlink was able to reduce the median latency to the same level as in
2023, while maintaining the minimal latency that was achieved in 2023.

We observed a strange behavior in the latency that we call the
"two-bell-pattern." It describes the behavior of serving two ranges
of latencies especially well. First lower latencies ($\approx$~80~--100~ms) and
second higher latencies ($\approx$~150~--250~ms) with a major gap in between.
The pattern appear among most countries, becoming even more apparent in 2024
compared to 2022. The cause of this pattern is unclear. We assume that there
is a specific set of variables causing Starlink terminals to fall in specific
categories causing higher or lower latencies. Those might include the age of
the hardware, wheather, time of usage, and more.

Additionally, we wanted to find out about diurnal variation. Looking at data
from April~2024, we found that there is no difference between the individual
week days. However, there is a strong variation between the hours of the day.
For example, one can expect lower latencies at night compared to during
lunchtime (times refer to UTC).

We also looked at packet loss and noted a high variation by country. Countries
such as the Philippines have packet loss ratios of up to 18\%, while Czechia
and Chile achieve less than 0.25\%. On the other hand side, central European
countries like Germany and the Netherlands experience high packet loss ratios.
Overall, most countries have a packet loss of 1 to 4\%.

Finally, we conclude that Starlink currently offers a good and stable
performance, even if it cannot compete with terrestrial internet connection.
Also, one should be aware that Starlink did not yet reach its full potential as
we saw a trend over the last couple of years. Infrastructure will likely
extend offering better performance in countries that currently show weak
results. Additionally, research still lacks a model for ideal satellite
networks. Such a model would describe the placement of satellites, ground
stations, as well as ideal routing.

\subsubsection*{RQ 2: Do latency and packet loss correlate?}

We expected an interaction in packet loss and latency and therefore correlated
them. We split the values by year and by country. The results were
inconclusive, which means we were not able to find a conclusion between latency
and packet loss. We assumed that there are other parameters influencing the
relationsship between latency and packet loss that have not been researched. On
of them might have been the number of probes per country, but we were able to
eliminate it as a determinant of behavior. Other factors are the version of the
Starlink terminal, the weather, satellite hardware components, or measurement
destinations. Given the results, at this time we cannot give a conclusive
answer to the research question.

\subsubsection*{RQ 3: How do networked satellite systems route?}

To answer research question 3, we further looked at the routing of the Starlink
satellite system. First, we wanted to find out the number of hops a route
usually takes. We found that most routes take 10 to 14 hops. The histogram
depicting the hops also shows the mentioned "two-bell-pattern", which might be
related to the pattern we found for latencies.

We also observed the change in latency per hop and found that usually there is
a spike in latency in the second hop. Likely, the spike originates from routing
through the Starlink satellite constellation, which is not reflected in the
traceroute results. We also found consequent spikes in latency. However, in
mapping the hops to their predominant \ac{ASN}, we found that those spikes lay
outside the Starlink domain. Starlink has an opportunity of improving its
performance at this point, but likely other entities are actually responsible
for the spike.

Finally, we looked at the influence of solar magnetic storms on the Starlink
latency and found that there is no correlation (i.e., both random
variables are orthogonal). This is against popular belief that claims that
Starlink performance is highly dependent on low solar magnetic storm intensity.

Overall, we observe Starlink being good at providing reasonable latencies in
remote regions. On the other hand side, Starlink also suffers from high
variation, while we expected a highly deterministic system (most factors are
predictable). We were able to eliminate some of the plausible causes (against
the public belief), but were not able to find the actual cause of the
variation. However, we could find a pattern ("two-bell-pattern") that might
contribute to the understanding of Starlink.

\section{Future Work}

During our work, we faced several limitations that required more effort. The
first limitation were the probes that we did not hold ourselves, but which
originate from RIPE~Atlas. This might lead to corrupted data in individual
probes. Ideally, the probes are operated by the researchers themselves, but
doing this a costly and labour-intensive undertaking.


In this work, measurement data mainly originated from RIPE~Atlas and
Cloudflare~Radar. Further platforms also allow contributing more perspectives
and variables that can be correlated with the Starlink performance. Prominent
examples are OONI and M-Lab.

Further, some directions were left unexplored due to process constraints. The
two-bell-pattern is an interesting observation that needs confirmation and
explanation as it was not clear why it occurred. It should also be correlated
with the similar observation in number of hops.

A detailed look at the routing strategy of Starlink on its satellite
constellation and later in terrestrial internet is left out. Most likely,
Starlink cannot give away complete responsibility after a packet is handed to a
different AS, as we saw in the latency spikes outside the Starlink domain. At
the same time, a proper model for such a routing behavior is missing. Certain
parts, like the placing of ground stations, are known, but a complete model was
not yet found.

Mentioned in \Cref{sec:bent-pipes}, \ac{SNO}s largely rely on \ac{ISL}s.
Looking at the traceroute data, it is obvious that \ac{ISL}s are used for
supporting remote regions. In parallel, research discusses the impact of
\ac{ISL}s on the performance, especially their influence on routing and
latency. Their impact and ideal usage are questions that need to be addressed
and are a matter of future research.

Another interesting topic is about comparing officially supported regions of
Starlink and where it is actually used. Due to various reasons (e.g., sanctions
or remote region), it cannot provide service in a specific set of countries
(e.g., Iran and North Korea). At the same time, users can still buy Starlink
equipment and take it to unsupported regions. We know that the equipment will
still work (e.g., Kiribati was not supported at the time of writing, but had
traffic), but services like IPinfo will not respond with the true location. It
is in question to what extent this is exploited and what use cases are.
