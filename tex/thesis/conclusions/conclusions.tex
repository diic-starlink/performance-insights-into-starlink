\chapter{Conclusions \& Outlook} \label{sec:conclusion}

In this thesis, we looked at the performance and resilience of the Starlink
satellite system. The main research questions are listed in
Chapter~\ref{sec:research-questions}.

To answer research question 1, we looked at the TLS handshake latency in the
individual years 2022~--~2024 and found that Starlink latency is at
approximately 80~ms median in 2024. However, it has to be noted that the
Starlink latency improved since 2022 in average, median, and minimal latency
(see Table~\ref{fig:weekday-statistics}).

We observed a strange behavior in the latency that we call the
"two-bell-pattern". It is visible in Figure~\ref{fig:latency-histogram-1} and
\ref{fig:latency-histogram-2}. It describes the behavior of serving two ranges
of latencies especially well. First lower latencies ($\approx$~80~--100~ms) and
second higher latencies ($\approx$~150~--250~ms) with a major gap in between.
It is unclear what the cause of this pattern is.

Additionally, we wanted to find out about diurnal variation. Looking at data
from April~2024, we found that there is no difference between the individual
week days (see Figure~\ref{fig:latencies-per-weekday}). However, there is a
strong variation between the hours of the day. For example, in the night, one
can expect lower latencies compared to lunchtime.

We also looked at packet loss and noted a high variation by country. Countries
like the Philippines have packet loss ratios of up to 18\%, while Czechia and
Chile achieve less than 0.25\%. However, also highly modernized countries like
Germany and the Netherlands experience in high packet loss ratios (see
Table~\ref{fig:packetloss-total}). In the end, for most countries have a packet
loss of 1 to 4\%.

We expected a correlation in packet loss and latency. We correlated the two
values and separated by country. Table~\ref{fig:packetloss-latency-correlation}
shows the resulting correlation values, where 0 indicates orthogonal random
variables and 1 or -1 indicates a strong correlation. Some countries hold
values that allow to make a conclusion, but others do not. We assumed that the
number of probes played a role, but were able to eliminate it as a factor for
the behavior (see
Table~\ref{fig:packetloss-latency-number-probes-correlation}). The correlation
might be related to currently unknown contributing factors that need to be
taken into consideration before drawing a conclusion. Therefore, we cannot make
a statement about the correlation of latency and packet loss. Research question
2 can consequently not be answered by the research.

To answer research question 3, we further looked at the routing of the Starlink
satellite system. First, we wanted to find out the number of hops a route
usually takes. We found that most routes take 10 to 14 hops (see
Figure~\ref{fig:hops-per-measument}). The histogram for the hops is also
showing the mentioned "two-bell-pattern". Both observations might be related,
which was not further researched in this thesis.

We also observed the change in latency per hop (i.e., what latency does the
traceroute measurement report in each hop). We found that usually there is a
spike in latency in the second hop. Likely, the spike originates from routing
through the Starlink satellite constellation, which is not reflected in the
traceroute results (see Figures~\ref{fig:latency-change-per-hop-1} and
\ref{fig:latency-change-per-hop-2}). We also found consequent spikes in
latency. However, mapping the hops to their predominant ASN, we found that
those spikes lay outside the Starlink domain. Starlink has an opportunity of
increasing their performance at this point, but likely other entities are
actually responsible for the spike.

Finally, we looked at the influence of solar magnetic storms on the Starlink
latency and found that there is no correlation (i.e., both random
variables are orthogonal).

\section{Future Work}

During our work, we faced several limitations that require more effort. The
first limitation were the probes that we did not hold ourselves, but originate
from RIPE~Atlas. That means unknown people in the world set them up as they
wished. This might have corrupted the data. Ideally, the probes are operated by
the researchers themselves, but this a costly and tedious thing to do.

Further, we did not explore quite some directions due to time constraints. The
two-bell-pattern is an interesting observation that needs confirmation and
explanation as it was not clear why it occurred. It should also be correlated
with the similar observation in number of hops.

Also, we did not look into detail how Starlink routes packets on their
satellite constellation and later on in terrestrial internet. It is clear that
Starlink cannot give away complete responsibility after a packet is handed to a
different AS. However, it is unclear what a proper model for such a routing
behavior would be.

Mentioned in Chapter~\ref{sec:bent-pipes}, \ac{SNO}s largely rely on \ac{ISL}s.
However, researchers still discuss the impact of \ac{ISL}s. It should be
researched what impact they have and how their ideal usage would work.
