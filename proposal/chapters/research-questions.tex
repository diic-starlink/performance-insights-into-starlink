\section{Research Questions} \label{sec:research-questions}

\begin{mdframed}
	\textbf{RQ: How do networked satellite systems perform in terms of latency and packet loss?}
\end{mdframed}

Previous work \cite{DBLP:conf/imc/MichelTGB22, DBLP:conf/infocom/MaCZCML23, Segan2020} showed first results for \ac{RTT} (i.e., latency), throughput, and packet loss. However, networked satellite systems are a cutting edge technology that undergo changes frequently. For example, \cite{DBLP:journals/corr/abs-2403-13497} et al. showed a heavily different performance for moving vehicles.
Therefore, it is worth taking a look if significant changes occurred. First, we will have a look on the current performance of networked satellite systems regarding latency and packet loss. Eventually, a system displaying current Starlink performance trends could be developed.

Additionally, we will analyze the Starlink user terminal firmware to evaluate resilience and privacy of the system.

\begin{mdframed}
	\textbf{RQ: How do networked satellite systems perform efficient routing?}
\end{mdframed}

In theory, satellites should integrate well in the existing internet architecture. However, it proved not to be the case. Especially high packet loss \cite{DBLP:conf/infocom/MaCZCML23} poses a problem as a continuous connection cannot be guaranteed.
It should be determined how networked satellite systems currently route packets. That includes considering bent~pipes and ISLs \cite{Hauri2020}.

Ideally, the firmware also allows assessing whether Starlink guarantees privacy and safe data transfer to modern standard of security.
Also, we will explore the limitations of fulfilling the previously mentioned standards in networked satellite systems including \ac{ISLs} and bent pipes.

\begin{mdframed}
	\textbf{RQ: What does the Starlink user terminal firmware reveal about the system's resiliency and privacy?}
\end{mdframed}

The user terminal firmware is a key factor for privacy and resiliency of the Starlink networked satellite system.
Previous attempts in accessing the terminal's built-in firmware were highly successful, but the firmware was not analyzed in detail. % TODO: Add citation

This thesis wants to explore that topic by describing a way to acquire the firmware, describe its structure, and analyze it in regard to resiliency and privacy.

