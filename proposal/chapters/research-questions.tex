\section{Research Questions} \label{sec:research-questions}

\begin{mdframed}
	\textbf{RQ: How do networked satellite systems perform in terms of latency and packet loss?}
\end{mdframed}

Previous work \cite{DBLP:conf/imc/MichelTGB22, DBLP:conf/infocom/MaCZCML23, Segan2020} showed first results for \ac{RTT} (i.e., latency), throughput, and packet loss. However, networked satellite systems are a cutting edge technology that undergo changes frequently. Therefore, it is worth taking a look if significant changes occurred. In the current measurement setup, we only measure \ac{RTT} and packet loss.

\begin{mdframed}
	\textbf{RQ: How does the relative position of a satellite to a user influence the performance?}
\end{mdframed}

The distance between a user's antenna and a satellite does not remain always the same. Over the course of a connection with a single satellite, the distance is constantly changing as the satellite is moving asynchronously with the earth (i.e., the relative position of satellite to antenna changes).
One hypothesis states that the strength of a signal is stronger when the users are in the center of a satellite's "beam".
On the other hand side, the signal is expected to be weaker at the "fringes" of the beam. It should be determined whether there is an actual difference and if the difference is significant.

\begin{mdframed}
	\textbf{RQ: How do networked satellite systems perform efficient routing?}
\end{mdframed}

In theory, satellites should integrate well in the existing internet architecture. However, it proved not to be the case. Especially high packet loss \cite{DBLP:conf/infocom/MaCZCML23} poses a problem as a continuous connection cannot be guaranteed.
It should be determined how networked satellite systems currently route packets. That includes considering bent~pipes and ISLs \cite{Hauri2020}.
Furthermore, versions of Starlink's user terminal firmware were obtained, even allowing root access to the user terminal \cite{QuarkslabsBlogUserTerminalFirmsware, WoutersDumpingUserTerminalFirmware}.
The firmware should be analyzed to determine how Starlink performs routing at the moment.
