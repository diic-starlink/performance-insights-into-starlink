\section{Motivation} \label{sec:motivation}

The internet plays a major role in our everyday's life. There are many technologies available that allow to connect to the web addressing different use cases (e.g., Wi-Fi for local area networks \cite{Henry2002} or \ac{FTTH} for high-speed wired connections to an end user's home \cite{Aleksic2010}).
However, there are several problems with such a setup. First, it requires wired infrastructure, which is quite expensive for networks in the wider area. For single end users, that is not affordable. Second, should the wired infrastructure be in place, it is vulnerable to attacks or natural catastrophes that render the service unusable.

Another emerging technology uses satellites as data transmission node. It is very resilient to human attacks or natural catastrophes as satellites are largely inaccessible. Therefore, governments installed dedicated satellite constellations to maintain communication in any given scenario. A satellite constellation is a group of artificial satellites that serve a specific purpose. Prominent examples of governmental satellite constellations are \textit{Beidou} and \textit{Galileo}.

Aside from crisis intervention, also businesses discovered opportunities. Providing communication over satellite allows users mobile access to information that usually require complex infrastructure. Users can access services like geographical data, radio frequencies, and even web access.
Especially the demand for web access is growing, while companies failed at providing acceptable latencies in the past.

However, also the demand for low latency access grew. Practice showed that \ac{GEO} satellites (35'786 km altitude) were not able to achieve the desired latency. However, \ac{LEO} satellites (200 - 2000 km altitude) were able to provide the desired latency. On the other hand side, \ac{LEO} satellites cover a smaller part of the earth's surface, which required more satellites to cover the same area.

\begin{figure}
	\label{fig:satellitegrowth}
	\centering
	\includegraphics[width=\textwidth]{./chapters/img/satellite-growth.png}
	\caption{Growth of satellite numbers of different satellite constellations since 2000. Note the logarithmic scale.}
\end{figure}

Therefore, different satellite communication providers (e.g., \textit{Starlink} or \textit{OneWeb}) started constructing their own networked satellite systems.
For different satellite constellations, the development of satellite numbers is shown in Figure~\ref{fig:satellitegrowth}. The numbers originate from N2YO \cite{N2YO2024}, a platform for tracking satellites.
It is visible that Starlink has far more satellites than any of its competitors. While Starlink arrives at more than 5500 satellites, its closest competitor, OneWeb, does not even reach 1000 satellites in 2024.

Aside from many earlier business failures \cite{Chan2002, Barboza2000}, it is still in question how well networked satellite systems actually work and whether they offer practical advantages. Previous work reported competitive latencies for \ac{LEO} systems, accompanied by increased packet loss \cite{DBLP:conf/imc/MichelTGB22}.
Also, it is in question whether networked satellite systems integrate well with existing protocols.

