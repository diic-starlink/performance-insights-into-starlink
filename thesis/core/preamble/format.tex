% This file includes all of the code that is used to format the thesis.
% Some packages are included if they are needed. This is done in the respective part and not at the beginning of this file.
%
% This file contains the following parts:
%   • Language an Character Set
%   • Penalties
%   • Indentation
%   • Footnotes
%   • Colors
%   • Size and Position of the Text Body
%   • Position of the Head and the Foot
%   • Margin Position and Width
%   • Header and Footer Format
%   • Caption Format
%   • Part Format
%   • Chapter Format
%   • Table of Contents


%%%%%%%%%%%%%%%%%%%%%%%%%%%%%%%%
%% Language and Character Set %%
%%%%%%%%%%%%%%%%%%%%%%%%%%%%%%%%

\usepackage[utf8]{inputenc} % Allows to input UTF8 characters.
\usepackage[T1]{fontenc}    % Allows to print special characters correctly.
\usepackage
[
    ngerman,         % German is used for the German abstract.
    main = american, % This is the main language of the thesis.
]
{babel}                     % Is responsible for sensible hyphenations.

% The following two commands make it such that the LaTeX compiler of Overleaf produces a PDF from with ligatures and mathematical symbols can be copied correctly.
% Also refer to: https://tex.stackexchange.com/questions/64188/what-are-good-ways-to-make-pdflatex-output-copy-and-pasteable
\input glyphtounicode
\pdfgentounicode=1


%%%%%%%%%%%%%%%
%% Penalties %%
%%%%%%%%%%%%%%%

\widowpenalties 2 10000 0


%%%%%%%%%%%%%%%%%
%% Indentation %%
%%%%%%%%%%%%%%%%%

\usepackage{calc} % Makes it easer to do math with TeX measurements.

\newlength{\myparindent}
\newlength{\myparskip}
\setlength{\myparindent}{1 em}
\setlength{\myparskip}{0 em}

\setlength{\parindent}{\myparindent}
\setlength{\parskip}{\myparskip}
\setlength{\parskip}{0 pt plus 1 pt minus 0 pt}


%%%%%%%%%%%%%%%
%% Footnotes %%
%%%%%%%%%%%%%%%

% Remove the footnote rule.
\setfootnoterule{0 cm}

% The footnote number is made bold and not in superscript.
\deffootnote[1.2 em]{1.2 em}{0 em}{\makebox[1.4 em][l]{\textbf{\thefootnotemark}}}

% The footnote number will not be reset after every chapter.
\makeatletter%
    \@removefromreset{footnote}{chapter}%
\makeatother


%%%%%%%%%%%%
%% Colors %%
%%%%%%%%%%%%

\usepackage[dvipsnames]{xcolor} % Allows it to define colors. The option says that common names can be used.

% Dark blue.
\definecolor{stroke1}{HTML}{2574A9} % This color is used as the standard color to highlight things.


% Coloring various different labels.
\colorlet{captionlabel}{black}
\colorlet{footerpagenr}{black}
\colorlet{footerchapter}{stroke1}
\colorlet{footerchaptername}{black}
\colorlet{footersection}{stroke1}
\colorlet{footersectionname}{black}
\colorlet{chapternumber}{stroke1}


%%%%%%%%%%%%%%%%%%%%%%%%%%%%%%%%%%%%%%%%
%% Size and Position of the Text Body %%
%%%%%%%%%%%%%%%%%%%%%%%%%%%%%%%%%%%%%%%%

% The new paper dimensions that are exclusively used.
\newlength{\mypaperwidth}
\setlength{\mypaperwidth}{210 mm}

\newlength{\mypaperheight}
\setlength{\mypaperheight}{297 mm}

% The text area uses aesthetically pleasing measurements in the same ratio as the page.
% These dimensions are always used, as the text area should be the same in the printed and digital version of the thesis.
\newlength{\mybodywidth}
\setlength{\mybodywidth}{140 mm}

\newlength{\mybodyheight}
\setlength{\mybodyheight}{198 mm}

\newlength{\myoutermargin}
\ifprintVersion
    \ifprofessionalPrint
        \setlength{\myoutermargin}{(\mypaperwidth - \mybodywidth) / \real{1.5} + \extraborderlength}
    \else
        \setlength{\myoutermargin}{(\mypaperwidth - \mybodywidth) / \real{1.5} - \mybindingcorrection}
    \fi
\else
    \setlength{\myoutermargin}{(\mypaperwidth - \mybodywidth) / \real{1.5}}
\fi

\newlength{\mytopmargin}
\setlength{\mytopmargin}{(\mypaperheight - \mybodyheight) / 3}
\ifprintVersion
    \ifprofessionalPrint
        \setlength{\mytopmargin}{(\mypaperheight - \mybodyheight) / 3 + \extraborderlength}
    \fi
\fi

\newlength{\myinnermargin}
\setlength{\myinnermargin}{\mypaperwidth - \mybodywidth - \myoutermargin}
\ifprintVersion
    \ifprofessionalPrint
        \setlength{\myinnermargin}{\mypaperwidth + \mybindingcorrection + 2\extraborderlength - \mybodywidth - \myoutermargin}
    \fi
\fi

\newlength{\mybottommargin}
\setlength{\mybottommargin}{\mypaperheight - \mybodyheight - \mytopmargin}
\ifprintVersion
    \ifprofessionalPrint
        \setlength{\mybottommargin}{\mypaperheight + 2\extraborderlength - \mybodyheight - \mytopmargin}
    \fi
\fi


%%%%%%%%%%%%%%%%%%%%%%%%%%%%%%%%%%%
%% Position of the Head And Foot %%
%%%%%%%%%%%%%%%%%%%%%%%%%%%%%%%%%%%

\newcommand{\goldenratio}{1.618}

\newlength{\myheadsep} % Distance from the header to the body.
\setlength{\myheadsep}{\mytopmargin / \real{\goldenratio} / \real{\goldenratio} - 1 ex}
\ifprintVersion
    \ifprofessionalPrint
        \setlength{\myheadsep}{(\mytopmargin - \extraborderlength) / \real{\goldenratio} / \real{\goldenratio} - 1 ex}
    \fi
\fi

\newlength{\myfootskip} % Distance from the body to the footer.
\setlength{\myfootskip}{\mybottommargin / \real{\goldenratio} - 1 ex}
\ifprintVersion
    \ifprofessionalPrint
        \setlength{\myfootskip}{(\mybottommargin - \extraborderlength) / \real{\goldenratio} - 1 ex}
    \fi
\fi


%%%%%%%%%%%%%%%%%%%%%%%%%%%%%%%
%% Margin Position And Width %%
%%%%%%%%%%%%%%%%%%%%%%%%%%%%%%%

\newlength{\mymargininnersep} % Distance between the margin and the body.
\setlength{\mymargininnersep}{7 mm}

\newlength{\mymarginoutersep} % Distance between the margin and the paper border.
\setlength{\mymarginoutersep}{12 mm}
\ifprintVersion
    \ifprofessionalPrint
        \setlength{\mymarginoutersep}{12 mm + \extraborderlength}
    \fi
\fi

\newlength{\mymarginwidth} % Width of the margin.
\setlength{\mymarginwidth}{\myoutermargin - \mymargininnersep - \mymarginoutersep}

\newlength{\mymarginwidthwithinnersep} % Width of the margin.
\setlength{\mymarginwidthwithinnersep}{\mymarginwidth + \mymargininnersep}

\usepackage
[
    % In the printed version, we add an extra border to each side as well as the binding correction for the width.
    \ifprintVersion
        \ifprofessionalPrint
            paperwidth = \mypaperwidth + 2\extraborderlength + \mybindingcorrection,
            paperheight = \mypaperheight + 2\extraborderlength,
        \else
            paperwidth = \mypaperwidth,
            paperheight = \mypaperheight,
        \fi
    \else
        paperwidth = \mypaperwidth,
        paperheight = \mypaperheight,
    \fi
    textwidth = \mybodywidth,
    textheight = \mybodyheight,
    outer = \myoutermargin,
    top = \mytopmargin,
    headsep = \myheadsep,
    footskip = \myfootskip,
    marginparsep = \mymargininnersep,
    marginparwidth = \mymarginwidth,
%    showframe, % Use this option for debugging purposes in order to the an outline of all of the different parts of the page layout.
]
{geometry} % Used in order to define the dimensions of the page and its layout.


%%%%%%%%%%%%%%%%%%%%%%%%%%%%%%
%% Header and Footer Format %%
%%%%%%%%%%%%%%%%%%%%%%%%%%%%%%

\usepackage
[
%    draft, % Shows a lot of rules denoting the dimensions of the head and foot. Use this option only for debugging.
]
{scrlayer-scrpage} % Allows to adjust the definitions of the head and foot of a page.

% Remove all predefined styles.
\clearpairofpagestyles

%%%%%%%%%%%%%%%%%%%%%%%%%%%%%%
% Dimensions and formats are defined.

% Define the dimensions of the head and the foot. Since we want some information to appear in the margin, we extend the head and the foot by the respective lengths.
\KOMAoptions
{%
    headwidth = \textwidth + \mymarginwidthwithinnersep,%
    footwidth = \myoutermargin : \textwidth,%
}

% Defines the formats for the chapter and section titles in the marks of the head.
\renewcommand*{\chaptermarkformat}{\normalfont\sffamily\small\color{footerchaptername}}
\renewcommand*{\sectionmarkformat}{\normalfont\sffamily\small\color{footersectionname}}

% Displays the chapter names in the head of both odd and even pages.
\automark[chapter]{chapter}
% Replaces the chapter name to the head of right pages with the section name if a section is present.
\automark*[section]{}

%%%%%%%%%%%%%%%%%%%%%%%%%%%%%%
% The head is defined.

% Head for even pages.
% Puts ›Chapter‹ followed by the current chapter number.
\lehead%
{%
    \begin{minipage}[b]{\mymarginwidth}%
        \small\raggedleft\normalfont\textsf{\textbf{\color{footerchapter}\chaptername\ \thechapter}}
    \end{minipage}
}
% Put the title of the current chapter/section into the center of the head but push it to the border.
\cehead{\hspace*{\mymarginwidthwithinnersep}\parbox{\textwidth}{\raggedright\leftmark}}

% Head for odd pages.
\rohead%
{%
    % Check whether a section has already started or not.
    \Ifstr{\rightmark}{\leftmark}%
    {%
        \begin{minipage}[b]{\mymarginwidth}%
            \small\raggedright\normalfont\textsf{\textbf{\color{footersection}Chapter\ \thechapter}}%
        \end{minipage}%
    }%
    {%
        \begin{minipage}[b]{\mymarginwidth}%
            \small\raggedright\normalfont\textsf{\textbf{\color{footersection}Section\ \thesection}}%
        \end{minipage}%
    }%
}
\cohead{\hspace*{-\mymarginwidthwithinnersep}\parbox{\textwidth}{\raggedleft\rightmark}}

%%%%%%%%%%%%%%%%%%%%%%%%%%%%%%
% The foot is defined.

% Displays the page number in bold in the margin, aligned toward the center. Further, a blue line is drawn above number.
% The starred variant is used, since we want the format of the foot to also apply to the pagestyle ›plain‹.
\lefoot*%
{%
    \vspace*{1 ex}%
    {\color{stroke1}\rule{\myoutermargin - \mymargininnersep}{0.5 mm}}\\
    \begin{minipage}[b]{\myoutermargin - \mymargininnersep}%
        \raggedleft\normalfont\color{footerpagenr}\textbf{\thepage}%
    \end{minipage}%
}
\rofoot*%
{%
    {\color{stroke1}\rule{\myoutermargin - \mymargininnersep}{0.5 mm}}\\
    \begin{minipage}[b]{\myoutermargin - \mymargininnersep}%
        \raggedright\normalfont\color{footerpagenr}\textbf{\thepage}%
    \end{minipage}%
}


%%%%%%%%%%%%%%%%%%%%
%% Caption Format %%
%%%%%%%%%%%%%%%%%%%%

\usepackage{caption}
\captionsetup
{
    font = small,
    labelfont = {bf, sf, color = captionlabel},
    format = plain,
    singlelinecheck = off,
}


%%%%%%%%%%%%%%%%%
%% Part Format %%
%%%%%%%%%%%%%%%%%
\usepackage{tikz} % Used in order to draw the stylistic elements.

\newlength{\mytmpa}
\setlength{\mytmpa}{1 mm}
\newlength{\mytmpb}
\newlength{\mytmpc}

%%%%%%%%%%%%%%%%%
% The following code draws the outline for a ›part‹ of the thesis.
% This command is used before the name of the part is displayed. It is void, as the part is added via \partlineswithprefixformat.
\renewcommand*{\partformat}{}
% This command calls \partformat (#2) and displays the name of the part (#3).
\renewcommand*{\partlineswithprefixformat}[3]%
{%
    #2
    \thispagestyle{empty}
    \setlength{\mytmpa}{0.618\mypaperwidth}%
    \setlength{\mytmpb}{0.382\mypaperheight}%
    \ifprintVersion
        \ifprofessionalPrint
            \setlength{\mytmpa}{0.618\mypaperwidth + \mybindingcorrection + \extraborderlength}%
            \setlength{\mytmpb}{0.382\mypaperheight + \extraborderlength}%
        \fi
    \fi
    \begin{tikzpicture}[overlay, remember picture]%
        \node [inner sep = 0, outer sep = 0, anchor = north] at (current page.north west)%
        {%
            \begin{tikzpicture}[overlay, remember picture]%
            \draw[color = stroke1, line width = 0.7 mm] (\mytmpa, 0) -- (\mytmpa, -\mytmpb);%
            \end{tikzpicture}%
        };%
        \node (align) [align = right, below = \mytmpb - 2 ex, inner sep = 0, outer sep = 0, anchor = north west] at (current page.north west)%
        {%
            \hspace{\mytmpa}\hspace{0.5 em}\partname\ \thepart\\[1 ex]
            \color{stroke1}#3%
        };%
    \end{tikzpicture}%
}
% This command defines various parameters for the ›part‹ format.
\RedeclareSectionCommand%
[%
    font = \normalfont\Huge\sffamily,
    prefixfont = \normalfont\Huge\sffamily,
]
{part}


%%%%%%%%%%%%%%%%%%%%%
%%% Chapter Format %%
%%%%%%%%%%%%%%%%%%%%%

\usepackage{etoolbox}

\newbool{chapterHasANumber}
\newbool{chapterHasAStar}
\renewcommand*{\chapterlinesformat}[3]%
{%
    % Check whether \chapter of \addchap has been used.
    \Ifnumbered{#1}{\setbool{chapterHasANumber}{true}}{\setbool{chapterHasANumber}{false}}%
    % Check whether \chapter* or \chapter has been used.
    \Ifstr{#2}{}{\setbool{chapterHasAStar}{true}}{\setbool{chapterHasAStar}{false}}%
    % Check whether a normal \chapter or something else is used.
    \ifboolexpr{bool{chapterHasANumber} and not bool{chapterHasAStar}}%
    {%
        \begin{tikzpicture}[overlay, remember picture]%
            \node [right = \myinnermargin, below = \mytopmargin, inner sep = 0, outer sep = 0, anchor = north west] (numbernode) at (current page.north west)%
            {%
                \hspace{\myinnermargin}%
                \sffamily\fontsize{60}{60}\selectfont%
                \color{chapternumber}%
                \thechapter%
            };%
            \node [inner sep = 0, outer sep = 0, anchor = north west] at (numbernode.south west)%
            {%
                \begin{tikzpicture}[overlay, remember picture]%
                    \draw[color = stroke1, line width = 0.7 mm] (\myinnermargin, -1 ex) -- (\paperwidth, -1 ex);%
                \end{tikzpicture}%
            };%
            \node (align) [text width = \textwidth - 2 cm, align = right, right = \myinnermargin + \mybodywidth, inner sep = 0, outer sep = 0, anchor = east] at (numbernode.west)%
            {%
                #3%
            };%
        \end{tikzpicture}%
    }%
    {%
        \begin{tikzpicture}[overlay, remember picture]%
            \node [right = \myinnermargin, below = \mytopmargin, inner sep = 0, outer sep = 0, anchor = north west] (numbernode) at (current page.north west)%
            {%
                \hspace{\myinnermargin}%
                \sffamily\fontsize{60}{60}\selectfont%
                \color{white}%
                \thechapter%
            };%
            \node [inner sep = 0, outer sep = 0, anchor = north west] at (numbernode.south west)%
            {%
                \begin{tikzpicture}[overlay, remember picture]%
                    \draw[color = stroke1, line width = 0.7 mm] (\myinnermargin, -1 ex) -- (\paperwidth, -1 ex);%
                \end{tikzpicture}%
            };%
            \node (align) [align = left, right = \myinnermargin, inner sep = 0, outer sep = 0, anchor = south west] at (numbernode.south west)%
            {%
                #3%
            };%
        \end{tikzpicture}%
    }%
}
\RedeclareSectionCommand%
[%
    font = \color{stroke1}\normalfont\huge\sffamily,
    afterskip = 20 pt,
]
{chapter}


%%%%%%%%%%%%%%%%%%%%%%%%
%%% Table of Contents %%
%%%%%%%%%%%%%%%%%%%%%%%%

% Format the table of contents to have a ›plain‹ page style.
\BeforeStartingTOC[toc]{\pagestyle{plain}}
\AfterStartingTOC{\thispagestyle{plain}}